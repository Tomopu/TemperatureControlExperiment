\documentclass[a4j, titlepage]{jarticle}

\usepackage{amsmath}
\usepackage{amssymb}
\usepackage{geometry}
\usepackage{hyperref}

\title{2024年度 専攻科 2S特別演習 電気工学分野\\温度制御シミュレータをターゲットにした各種方式によるPIDパラメータ調整の試行}
\author{釧路工業高等専門学校 電子情報システム工学専攻 2年\\泉 知成, 河江 蒼生, 坂本 尊, 福島 祥太, 森 隆志}
\date{\today}

\begin{document}
\maketitle
\tableofcontents
\clearpage
\section{演習の目的}

\clearpage
\section{選択した調整方法での作業プロセス}
本実験では, 以下の$2$つの方法を使用して変数値の調整を行う.

\subsection{選択した調整方法}
\begin{itemize}
    \item Ziegler \& Nichols 限界感度法 (2.2)
    \item Ziegler \& Nichols ステップ応答法 (2.3)
\end{itemize}

\subsection{Ziegler \& Nichols 限界感度法の作業プロセス}
\begin{enumerate}
    \item Pのみを使用する様に設定する.(但し、I, Dは0に)
    \item 目標値を $100\,^{\circ}\mathrm{C}$, 比例帯 (PB) も始めは大きく設定する.
    \item 温度調節開始。
    \item 温度が持続振動状(波形的な曲線)になるまで比例帯 PB を変えて試行する.
    \item 持続振動状態になった時の周期 $T_c$ と、PB から比例ゲイン $K_{pc}$ を求める
    \item PID 制御のパラメータを以下の計算式で求める.
\end{enumerate}
\begin{align*}
    K_p = 0.6 * K_{pc} \\
    T_i = 0.5 * T_{c} \\
    T_d = 0.125 * T_{c}
\end{align*}

\subsection{Ziegler \& Nichols ステップ応答法の作業プロセス}
\begin{enumerate}
    \item ON/OFFを使用する様に設定する.
    \item 目標値を $100\, ^{\circ}\mathrm{C}$ に設定する.
    \item 温度調節開始.
    \item 温度の変化カーブを観測.
    \item 観測結果からムダ時間 $L$,勾配 $R$ を求める.
    \item PID 制御のパラメータを以下の計算式で求める.
\end{enumerate}
\begin{align*}
    P:\ K_p = \frac{1.2}{R * L}, \\
    I:\ T_i = 2 * L, \\
    D:\ T_d = 0.5 * L
\end{align*}

\section{計測および各種計算等の結果}
\section{結果}
\section{考察検討}
\section{役割分担}

\begin{thebibliography}{9}
  \bibitem{label} author, title, publish, 2023.
\end{thebibliography}

\end{document}